\documentclass[12pt]{article}
\usepackage{cite}
\usepackage[labelfont=bf,labelsep=period,justification=raggedright]{caption}
\usepackage{graphicx}
\usepackage{color}
\bibliographystyle{plos2009}
\setlength{\topmargin}{0.0cm}
\setlength{\textheight}{23cm}
\setlength{\textwidth}{16cm}
\setlength{\oddsidemargin}{0.5cm}
\setlength{\evensidemargin}{0.5cm}\setlength{\voffset}{-24pt}
\usepackage{fancyhdr} 
\pagestyle{fancy}
\lhead{} 
\chead{} 
\rhead{\bfseries regevol Manual} 
\lfoot{} 
\cfoot{} 
\rfoot{} 
\setlength{\parskip}{8pt}  
\usepackage[colorlinks]{hyperref}
\hypersetup{linkcolor=black,citecolor=black}
\usepackage{float}
\clubpenalty=10000
\widowpenalty=10000

\author{Kevin Bullaughey \\ \normalsize{kbullaughey@gmail.com}} 
\title{\texttt{regevscape} Manual}

\begin{document}
\maketitle

\section*{Overview}

This manual provides documentation of the \texttt{regevscape} program, which is short for \textbf{reg}ulatory \textbf{ev}olution land\textbf{scape} simulator. \texttt{regevscape} simulates sequence evolution of an enhancer using a computational model of regulatory function based on the Segal model\cite{Segal:2008fk}, a fitness penealization for misexpression, and forward population simulations. It was originally used by Bullaughey\cite{KLBullaughey11232010} to investigate nucleotide substitution processes in a regime of stabilizing selection for a particular regulatory output of simple toy enhancers. For an overview of the model and deatils on the implementation, please refer to the original paper\cite{KLBullaughey11232010}. The purpose of this manual is to document usage of the program and aid interpretation of the output.

\bibliography{refs}
\end{document}
